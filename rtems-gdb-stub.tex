%$Id: rtems-gdb-stub.tex,v 1.22 2008-07-31 00:56:59 till Exp $
\documentclass{article}
\usepackage{bookman}
\usepackage{html}

\newcommand{\cmd}[1]{{\tt #1}}

\newcommand{\nix}{}
\newcommand{\bul}{$\bullet$\hspace{.5em}}

% Have no clever way to set the link depending on
% the install location w/o using major
% 'configure/automake' tricks. Disable for now, sorry...
%
%\begin{htmlonly}
%\renewcommand{\nix}{
%	\htmladdnormallink{PDF version of this document}
%		{../../gdb/rtems-gdb-stub.pdf}
%}
%\end{htmlonly}

%*************************************************
%THIS IS STILL ALPHA / EXPERIMENTAL CODE
%*************************************************

\title{GDB Agent for RTEMS}

\author{Till Straumann $<$strauman$|$at$|$slac.stanford.edu$>$}
\date{3/14/2005; last modified 7/29/2008\\
\nix
}

\begin{document}
\maketitle
\tableofcontents

\section{Introduction and Features}

This document covers version 1.5 of the target agent.

\subsection{Architecture}

This gdb stub is an implementation of the remote end of the
GDB {\em remote protocol} for a RTEMS target. Unlike other stubs
available for RTEMS (e.g., the m68k implementation), this
version does NOT run at `exception' level but is intended to
be less intrusive. It's primary use is debugging of user-level
/ application code rather than the kernel itself (kernel
debugging is still possible to a certain extent but the stub
essentially needs a functional and initialized system to
operate).

The stub runs as a `daemon task' executing debugging commands
as demanded by the host gdb. Unlike a `low-level stub' which
interrupts/stops the target system as a whole while talking
to gdb, this implementation only affects the execution of
one or multiple `target threads' leaving other tasks alone.

As a consequence, it is possible to use TCP/IP as a communication
protocol (no breakpoints in networking must be set, though!).

\subsection{Multithreading}
Gdb semantics expect the target system to be `interrupted' or
`stopped' while listening to gdb commands. This implementation,
OTOH, attempts to be as little intrusive as possible. For this
purpose, a list of `stopped' tasks is maintained internally.
Tasks are interrupted only as a result of hitting a breakpoint,
incurring an exception or an explicit request by the debugger.
Interrupted tasks can be inspected as usual (stack frame,
register contents etc.) and can be `continued/resumed'.
Only `stopped' tasks can be inspected - the others continue
executing normally.

The stub also creates a special helper task context which is
always interrupted when the target is stopped, hence it is
possible to do non-task related
work (inspect/change memory, disassemble, ...) without having
to stop an application task.

\subsection{Cexp Support}
A patch to `gdb' is provided extending the remote protocol
in order to support loadable modules. Gdb can inquire the
list of currently loaded modules and their section addresses
as well as load/unload additional modules on the fly.

\subsection{Supported Targets}
While many parts of the stub are intended to be portable
and target-architecture independent, there are a few pieces of code
which are BSP/CPU dependent. Currently, low-level support
has been implemented for the i386/pc386, m68k/coldfire and
powerpc/shared (and derived - should probably work on any
`new-exception processing' ppc BSP) BSPs.

Adding support for a new target should not be extremely
hard. It mainly consists of writing two routines for
packing/unpacking register values into an array formatted
as expected by gdb.
Also, a low-level exception handler needs to be
written (mostly for mapping exception codes to signal numbers).
E.g., the i386/pc386 specific file consists of ~400 lines
of quite straightforward code.

\subsection{Supported Connection Methods}
One of the strengths of the gdb remote protocol is its simplicity
which makes it possible to use many character-stream based
I/O methods for exchanging messages.
Currently, the stub supports TCP/IP, serial port communication
via RTEMS' termios driver and plain ``dumb'' character devices
as long as they can be accessed via the file-system ({\tt /dev/xzy}).

\subsection{GDB Patches}
One important design goal was not having to patch gdb itself.
Unfortunately, it was not completely possible to meet this
goal. However, the necessary patches are believed to be
as small and as modular as possible.

\begin{itemize}
\item
 A small patch is needed to fix a bug [has been submitted
 as GDB bug \#2029
 but to-date, I don't know if it has been merged] in the
 powerpc stack unwinding.
\item
 A small patch is needed to make it possible to obtain
 the list of tasks running on the target without interrupting
 and stopping them all
 (which would result in violating design goal \#1: only
 interrupt tasks on request [explicit or breakpoint]).
\item
 An additional patch adds a new gdb target (``\cmd{rtems-remote}'')
 with protocol
 extentions to support loadable modules (CEXP). Note that
 the stub works fine without CEXP, however.
\end{itemize}

\noindent
All of these patches are usually available with the source-code
of the stub as a ``{\tt .diff}''-file of the format
{\tt gdb-<version>-<date>.diff}.

\section{Building}
For the prerequisites consult table~\ref{tbl:prereq}
\begin{table}
\begin{center}
%\begin{tabular}{l@{\hspace{3em}\bul\hspace{0.5em}}l}
%host:   &\begin{minipage}[t]{9cm}\raggedright
%	    download and unpack gdb-6.6 (or any later version for which a patch
%		has been developed -- check the \protect{\verb4gdb-<version>-<date>.diff4}
%		patches distributed with rtems-gdb-stub
%		\end{minipage} \\
%		&patches distributed with rtems-gdb-stub \\
%target: &(optional) libbspExt $>=$ version 1.3    \\
%		&(optional) CEXP $>=$ version 1.5.beta
%\end{tabular}
\begin{tabbing}
target:\hspace{4em}\=\bul\= \kill
host:  \> \bul \> download and unpack gdb-6.8 (or any later version \\
       \>      \> for which a patch has been developed -- check the \\
	   \>      \> \protect{\verb4gdb-<version>-<date>.diff4} patches distributed \\
	   \>      \> with rtems-gdb-stub \\
	   \> \bul \> patches distributed with rtems-gdb-stub \\
target:\> \bul \> (optional) libbspExt $>=$ version 1.3  \\
       \> \bul \> (optional) CEXP $>=$ version 1.5.beta
\end{tabbing}
\caption{Prerequisites}
\label{tbl:prereq}
\end{center}
\end{table}

\begin{itemize}
\item
  Change directory to the gdb source topdir, create a build
  subdirectory (name of your choice), chdir there and configure
  (besides `powerpc', the `i386' and 'm68k' CPUs are supported and the
  cross-debugger may be configured accordingly)%
\footnote{Older versions of the gdb patch (pre 6.8) required the target
to be {\tt <cpu>-ssrl-rtems}}:
\begin{verbatim}
>  mkdir build
>  cd build
>  ../configure --target=powerpc-rtems  
\end{verbatim}

  Compile by typing {\tt make} (gnumake) and install manually:
\begin{verbatim}
>  make
>  install -s gdb/gdb <target-dir>/powerpc-rtems-gdb
\end{verbatim}

  If you want to install everything including documentation you
  might need more elaborate {\tt configure} options to get the
  install locations and the executable name right.
  
\item
  The target agent is part of the {\tt ssrlApps} package and
  should be build automatically as part of that package.
  If you don't want to use libbspExt (on powerpc) or Cexp
  you have to redefine the respective {\tt USE\_xxx} variables
  in the {\tt Makefile}.

  Note that if you {\em do\/} use those packages then you must
  make sure that your application executes the respective initialization
  routines ({\tt bspExtInit()} and {\tt cexpInit()}, respectively;
  consult the header files for any arguments and/or return value).

  If you try to build {\em rtems-gdb-stub} independently from
  {\tt ssrlApps}, an additional step is required because
  the agent needs a private header file, {\em cexpmodP.h},
  from Cexp (only if Cexp support is to be used) that is {\em not}
  installed with the Cexp package: Edit the Makefile
  and point the {\tt CEXP\_SOURCE\_PATH} variable to the
  correct location.

\item
  As with any RTEMS application, you must not forget to link all
  required RTEMS managers -- otherwise you incur what seem to be
  obscure errors. If you have plenty of memory then simply including
  all managers by setting
  {\tt MANAGERS=all} in your {\tt Makefile} is most comfortable.
  
\end{itemize}

\section{Debugging Session}

\subsection{Prerequisites}
\subsubsection*{Compilation}
	Code to be debugged must be compiled with the `{\tt -g}' option.
    (preferrably everything)%
\footnote{Until recently, `{\tt -g}' was implicitely set when building
RTEMS proper. Unfortunately, this (late Nov. '05) no longer seems to be
the case. I found that passing `\mbox{\tt RTEMS\_CFLAGS=-g}' on the command
line to the \cmd{configure} script when building RTEMS did the job. I
don't know if this is the recommended way, though. YMMV.
}.

	Optimization Caveat: It is possible to use GDB with optimized
	code (I do it all the time) but this may change the flow
	of execution, inline subroutine calls and may cause
	variables to `disappear' etc.

\subsubsection*{Front-End}
	You want to use a GUI frontend together with cross-gdb, e.g.,
\begin{verbatim}
		ddd --debugger <cross_arch>-rtems-gdb
\end{verbatim}

	{\em Note\/}: you need to set {\tt PATH} prior to starting ddd/gdb,
    see below.  

\subsubsection*{RTMF}
	GDB is a quite powerful program -- make sure you know how to
	use it! The `rtems-remote' target only adds 2 new commands.
	Everything else is off-the shelf GDB.
	Note that not all available commands necessarily make sense
	on this specific target.

\subsubsection*{Path Settings}
    Cross-debugging involves both, a host computer and the
    target system executing the actual code. The host computer
    needs access to {\em the same version} of the object files
	that the target computer is executing.

	Upon connection to the target, the host computer obtains
	a list of currently loaded object files from the target.
	In order for the host GDB being able to locate the necessary
	object files, the {\tt PATH} environment variable must point to
	the directories containing these objects. Note that GDB
	doesn't define a dedicated path variable for `cross-architecture
	objects' but requires the user to set the ordinary {\tt PATH}
	that is also searched for host executables!
	Note that {\tt PATH} is an environment variable and cannot be
	changed from within a GDB session (GDB's \cmd{environ} and \cmd{path}
	commands only affect the {\tt PATH} as seen by a native debuggee
	and don't have any effect in a cross-gdb session).

	GDB also features a \cmd{load} command to instruct the target
	computer to load/link object modules. Since the target
	computer usually sees a directory tree that is different
	from the host, any path information is discarded from the
	\cmd{load} argument -- instead, the target's {\tt PATH} environment
	variable must contain the directory where the target can
	locate the object that is to be loaded.

\subsection{Starting the Debugging Agent}
	On the target, the debugging agent must be running.
	It needs to be linked into the system and is started
	by calling
\begin{verbatim}
  int rtems_gdb_start(int agent_prio, char *dev_name);
\end{verbatim}

	Passing a priority of 0 lets the daemon pick its default
	scheduling priority. The second argument defines the
	connection method to be used. A NULL pointer lets the
	daemon listen on TCP port 2159 (registered port number for the
	gdb remote protocol) for an incoming connection,
	if a string is passed, it must be the path to a character
	device, e.g., {\tt /dev/ttyS1}.

	The agent can be stopped (\verb|rtems_gdb_stop()|) and restarted
	with a different priority and/or communication method.

	{\em Note\/}: rtems 4.6.2 requires a patch for this to work safely
	since the operation involves one task closing a socket
	on which another task is blocking.

\subsection{Connecting to the Target}
	gdb can use either serial-port, TCP or UDP connections.
	Note that either uses certain resources on the target (e.g.,
	the termios driver or the TCP/IP stack, respectively)
	and a debugging session may deadlock the system if
	the debugging agent needs a resource that is locked
	when the debuggee runs into a breakpoint.

	The syntax for connecting to a target is (TCP)
\begin{verbatim}
		(gdb) target rtems-remote <target>:2159
\end{verbatim}
	or (serial)
	 -- note that the agent's serial port runs at
	115200 8N1 -- use GDB's \cmd{remote baud} command:
\begin{verbatim}
		(gdb) target rtems-remote <com_port_on_host>
\end{verbatim}

	UDP connections can also be used but gdb warns against
	them being less reliable. There is usually no point
	using UDP if TCP is available. One important exception
	would be if the target features a ``standalone'' UDP
	communication channel (e.g., a simple ethernet/IP/UDP
	packet driver/stack; such a driver/stack could be
	used to debug TCP/IP itself) or has reduced networking
	capabilities.
\begin{verbatim}
		(gdb) target rtems-remote udp:<target>:2159
\end{verbatim}

	Note that \cmd{rtems-remote} is an extension of the
	standard GDB `remote' target -- it adds support
	for Cexp modules (names, section addresses etc.).

	Once connected to the target, the debugger is
	attached to a dummy thread context. All user and
	system threads are executing normally at this point.

	Use the \cmd{detach} command to close the connection
	to the target at any time. All stopped threads will
	be resumed.

\subsubsection{Sharing the Console Serial Line}
	If a priority less than zero is passed to {\verb|rtems_gdb_start()|}
    the agent is executed in the context of the the caller, i.e.,
    no separate task is created. Thus, the agent may be run from
    a shell (e.g., ``Cexp'') and take over the console line for
    the duration of the session, i.e., until the \cmd{detach}
    command is issued from GDB. In this ``foreground'' mode
    of operation, the agent terminates after a the session is ended
    returning control to the caller of {\verb|rtems_gdb_start()|}.

	A sample session example:
\begin{verbatim}
	    Cexp>rtems_gdb_start(-1,0)
	    GDB Daemon starting
	< scrambled output since terminal mode changed >
	    .y
	< exit from terminal program and start gdb >
	    (gdb) target rtems-remote < /dev/ttySxx >
	    (gdb)    < session commands >
	    (gdb) detach
	    (gdb) quit
	< restart terminal program and resume Cexp session >
	    Cexp>
\end{verbatim}

\subsection{Switching Thread Context}
	In order to inspect a thread's stack, registers etc.
	it must be stopped.
	A thread is stopped either because it runs into a
	breakpoint or if the debugger is explicitely
	attached to it (\cmd{thread} command). If you want to attach
	to a specific thread you need to know its `id'
	(i.e., GDB's id which is orthogonal to the RTEMS task id
	-- use \cmd{info threads} to obtain a task list).

	Once stopped, a thread remains suspended until you
	issue the \cmd{continue} command or \cmd{detach} (terminate
	the session). Either of those operations lets all stopped
	threads resume.

	You can hit \cmd{$<$Ctrl$>$-C} to interrupt the target; this
	only interrupts the `dummy'/`helper' thread and
	attaches GDB to it.  I.e., 
\begin{verbatim}
		target rtems-remote xx:2159 //connect, attach to 'helper'
\end{verbatim}
	At this point, the helper is stopped - you can inspect its
	registers etc. everyone else continues normally
\begin{verbatim}
		info threads
		t 14             //switch to thread ID 14
\end{verbatim}
	Thread 14 is stopped. The helper remains stopped. Inspect
	T14's registers + stack
\begin{verbatim}
		t 12             //switch to thread ID 12
\end{verbatim}
	helper, T14 and T12 are stopped; inspect T12's registers + stack
\begin{verbatim}
		t 14             //switch back to thread ID 14
\end{verbatim}
	helper, T14 and T12 remain stopped; inspect T14's registers + stack
\begin{verbatim}
		cont             //resume everything
\end{verbatim}
	helper, T14 and T12 are resumed
\begin{verbatim}
		<Ctrl>-C         //interrupt
\end{verbatim}
	interrupt, stop helper and attach to it - same state as
	after connection was established.

\subsection{Executing Subroutines on the Target}
When evaluating expressions at the \cmd{(gdb)} command line (e.g., within
a \cmd{call} or \cmd{print} command) gdb may need to execute a subroutine
on the target. For example,
\begin{verbatim}
		(gdb) print printf("Hello\n")
\end{verbatim}
invokes the \cmd{printf} routine on the target, retrieves the return value
and prints it to the \cmd{gdb} console.

This is actually a pretty complex operation which involves creating a dummy stack
frame and appropriate arguments etc. on the target. Some targets' exception
handlers may not expect modifications of the stack and might crash as a result
of an attempt to call a routine on the target.

Also, it should be noted that the routine is always executed in the
context of the ``current task''. Hence, the ``current task''
\begin{itemize}
\item must have sufficient stack space.
\item must provide an appropriate execution environment. E.g., if the
      current task is an ``integer-only'' task then an attempt to invoke
      math routines may fail (crash).
\item must not have been suspended due to an exception condition (i.e.,
      it must not be a ``dead'' or ``killed'' thread).
\end{itemize}
The `helper' task provides a suitable environment for the purpose of
executing subroutines on the target.

\subsection{DDD Note}
After reloading any object on the target (using the \cmd{rtems load} or
\cmd{rtems sync-objs} commands) the source code displayed
by DDD may not be current (since the commands were performed
at the GDB level, unnoticed by \cmd{ddd}).
Use \cmd{ddd}'s \cmd{Reload Source} menu entry from the
\cmd{Source} pull-down menu or the equivalent accelerator key combination
\verb|<Ctrl>-<Shift>-L| (default) to refresh the display.

\section{RTEMS Specific GDB Commands}
\begin{itemize}
\item
use \cmd{rtems load} instead of \cmd{load}.

\item
use \cmd{target rtems-remote} instead of \cmd{target remote}.
With statically linked applications the plain \cmd{remote}
target is enough -- the enhanced target version \cmd{rtems-remote} has added
support for loadable modules: it queries the target for
a list of loaded objects and their section addresses.

\item
\cmd{rtems sync-objs} (re)-synchronize gdb's file and section information
	  with what's current on the target. Use this command if you
	  added, removed or re-loaded modules from the Cexp$>$ prompt
	  instead of using \cmd{rtems load} from inside GDB.

\item
\cmd{info threads} semantics have slightly changed. Only a list
	  of thread IDs/names is retrieved but not their stack frame
	  or IP address (impossible without stopping all threads).
\end{itemize}

{\em Note that even if you do not use Cexp or the {\tt rtems-remote}
target you still must use a patched {\tt gdb} in combination with this
stub. One (small but important) change the patch achieves is avoiding
to attach to all of the tasks running on the target  (and thus stop them) 
which would bring the session to a grinding halt.}

\verb|<<TODO>>|

	More details; more commands; more implementation specifica

\verb|<</TODO>>|
	

\section{Example Session}
	Assume you compiled a module in your home directory
\begin{verbatim}
		cross-gcc -g -c blah.c -o /home/john/blah.o
\end{verbatim}
    You copy the object to a TFTP server where it can be
	`seen' by the target.
\begin{verbatim}
		cp blah.o /tftpboot
\end{verbatim}
	Prior to starting a debugging session, the {\tt PATH} on the
	host is set up:
\begin{verbatim}
		setenv PATH $PATH":/home/john"
\end{verbatim}
	On the target, (e.g., via startup script or from the Cexp$>$
	prompt) the PATH is also setup:
\begin{verbatim}
		Cexp> setenv("PATH","/TFTP/BOOTP_HOST/:<more dirs>:",1)
\end{verbatim}
	Start a GDB session on the host
\begin{verbatim}
		powerpc-rtems-gdb
\end{verbatim}
	Connect to the target
\begin{verbatim}
		(gdb) target rtems-remote <target>:2159
		// gdb obtains object file list and searches
        // PATH for 'blah.o'
\end{verbatim}
	Recompile `blah.c'
\begin{verbatim}
		cross-gcc -g -c blah.c -o /home/john/blah.o
		cp blah.o /tftpboot
\end{verbatim}
	Reload module from GDB prompt
\begin{verbatim}
		(gdb) rtems load blah.o
\end{verbatim}
\end{document}
